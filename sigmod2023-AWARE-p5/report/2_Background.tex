
\section{Background and Overview}
\label{sec:background}

This section reviews the main characteristics of CLA
and its limitations, which directly motivate key design decisions of \name.
% Inspired by database compression schemes and sparse matrix representations,
CLA~\cite{ElgoharyBHRR16, ElgoharyBHRR18} is a lossless compression framework leveraging
column-wise compression, column co-coding (encode groups of columns as single units), and heterogeneous column encoding schemes.
This design exploits characteristics of feature matrices---with categorical and numerical features in columns---namely,
tall and skinny matrices, non-uniform sparsity, low cardinality and correlations.
Since the selection of column groups is strongly data-dependent,
CLA introduced a sampling-based compression planning for online compression after
an initial read of an input matrix.
Once compressed, specialized kernels work directly on the compressed representation if applicable and efficient,
otherwise fall back to decompression followed by uncompressed kernels.

\subsection{Limitations of CLA}


Despite CLA's compelling properties---allowing operations directly on the compressed representations---there are limitations and missing functionality that hinder general applicability.

\textbf{Compression Costs:}
The original CLA's \cite{ElgoharyBHRR16} sampling-based compression planning used a sample fraction and a co-coding algorithm
that---ignoring greedy partitioning---required $\mathcal{O}(m^3)$ group extractions from the sample.
The refined CLA \cite{ElgoharyBHRR18} %\redd{increased the sample fraction to $f=0.05$ for better robustness, but}
improved the co-coding approach via memoization to $\mathcal{O}(m^2)$ group extractions.
However, despite column classification into compressible and incompressible columns,
matrices with no or minor compression benefits were recognized much too late,
after incurring already substantial overhead.
Furthermore, the super-linear co-coding complexity becomes infeasible for millions of columns.

\textbf{Compressed Intermediates:}
CLA performs online compression for input matrices and keeps outputs of amenable operations like matrix-scalar element-wise multiplications,
or scaling and shifting compressed.
Other operations produce uncompressed outputs (e.g., after feature transformations or data cleaning).
In complex, exploratory ML pipelines there are multiple sources of redundancy though,
which would largely benefit from a more fine-grained selection of intermediates and optimization of their individual compression schemes.

\textbf{Optimization Objective:}
The design and implementation of CLA aimed at the sweet spot of compressed operation performance close to uncompressed (low in-memory overhead),
while achieving good compression ratios (fit larger data in memory, reduced I/O for large data).
Besides this hand-crafted tuning, the internal objective for selecting compression schemes and co-coding then only focuses on minimizing compressed size.
Fundamentally, however, the overall optimization objective should be total execution time,
factoring in compression, compressed operations, and potential I/O in order to better adapt to data, operation, and cluster characteristics.



% CLA
% DDC, OLE, RLE, UC. Encodings: DDC 2, OLE 1, RLE 1, UC 1. -> 4(5)

% NEW:
% DDC 4, CON 2, OLE, RLE, UC, SDC 4 * 2 * 4, PFOR 2 * 4 = 4 + 2 + 1 + 1 + 1 + 32 + 8 = ColumnGroup Index Structures 49
% Dictionary encodings: 3 (Double Dense, MatrixBlock Dense, MatrixBlock Sparse) * 49 = 147 possibilities.
% with UByte (debatable)
% DDC 5, CON 2, OLE, RLE, UC, SDC 5 * 2 * 4, PFOR 2 * 5 = 5 + 2 + 1 + 1 + 1 + 35 + 10 = ColumnGroup Index Structures 55
% Dictionary encodings: 3 (Double Dense, MatrixBlock Dense, MatrixBlock Sparse) * 55 = 165 possibilities.

% DDC 7 * 2*( FOR), CON 2, UC, SDC 7 * 4 * 3 (Zero,_, FOR)  SDCSingle, 4*2 
% 7 *2 + 2 + 1 + 1 + 1 + 7*4*3 + 4*2 = 109 col group index structures.
%  3 * 111 = 333

% YetAgain:
% DT = 3 // dictionary types
% DDC 7 * DT  = 21
% DDCFOR 7 * DT =21
% CON  DT = 3
% Empty 1  = 1
% Uncompressed = 2 // Sparse or dense 
% SDC 7 * 4 * DT = 84
% SDC Zero 7 * 4 * DT = 84
% SDC Single 4 * DT = 12
% SDC Single Zero 4 * DT = 12
% SDC FOR 7 * 4 * DT = 84
% RLE DT = 3
%% Sum = 327

\begin{table}[!t]
    \centering \setlength\tabcolsep{10pt}
    \caption{\label{tab:diff}Differences of CLA and \name}
    \vspace{-0.4cm}
    \begin{tabular}{rcc}
        \toprule
                               & \textbf{CLA}~\cite{ElgoharyBHRR16,ElgoharyBHRR18} & \textbf{\name}   \\
        \midrule
        Co-Coding              & $\mathcal{O}(m^2)$                                & $\mathcal{O}(m)$ \\
        Column Group Encodings & 4 (5)                                             & 7 (327)          \\
        Materialization        & Eager                                             & Deferred         \\
        Optimization Objective & Data                                              & Data \& Ops      \\
        Matrix Multiplication  & MV, VM                                            & MV, VM, MM       \\

        \bottomrule
    \end{tabular}
    \vspace{0.3cm}
\end{table}


\subsection{\name\ Overview}

Our workload-aware compression addresses these limitations with new techniques for compression planning, compressed intermediates, and different optimization objectives. Table~\ref{tab:diff} highlights these key differences between CLA and \name.
%
First, to reduce compression time, we introduce a new co-coding technique that performs group \emph{combinations} instead of group \emph{extractions}, reducing the overhead to analyze groups. Our co-coding approach further includes a new enumeration heuristic that only evaluates $\mathcal{O}(m)$ group combinations.
%
Second, for extended utilization of compressed intermediates, we provide new column group encodings that facilitate shallow-copy operations. Table~\ref{tab:diff} shows the number of high-level column group types and---in parenthesis---the total number of variations of these encodings. We also introduce a deferred operation/encoding design, where compressed operations can output different types of encodings, allowing compressed intermediates where CLA would decompress or return inefficient representations. Furthermore, \name\ natively supports compressed matrix-matrix multiplication (even with two compressed inputs), unlike CLA that would process it via repeated matrix-vector multiplications and thus decompresses one side.
%
And above all, \name\ uses a cost-based optimization objective of minimizing the workload execution time and thus, tuning the compression process, the compressed representation, and compressed operations in a principled way.
