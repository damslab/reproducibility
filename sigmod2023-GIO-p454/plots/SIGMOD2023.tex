\documentclass[sigconf,nonacm,screen]{acmart}

\usepackage{filecontents}
\usepackage{textcomp}
\usepackage{pgfplots}
\usetikzlibrary{patterns}
\usepackage{ifthen}
\usepgfplotslibrary{groupplots}
\RequirePackage{keyval}
\usepackage{multirow}
\usepackage{multicol}
\usepackage{csvsimple}
\usepackage[utf8]{inputenc}
\newcounter{row}
\newcounter{col}
\usepackage{wrapfig}
\usepackage{textgreek}
\usepackage[inline]{enumitem}
\usetikzlibrary{matrix, positioning}
\usetikzlibrary{patterns,tikzmark}
\usetikzlibrary{matrix,decorations.pathreplacing,calc}
\usepackage{hf-tikz}
\usepackage{pifont}
\usepackage{subfig}
\usetikzlibrary{chains,fit,shapes}
\usetikzlibrary{arrows.meta,
    chains,
    positioning,
    shapes.symbols}
\usetikzlibrary{decorations,calligraphy}
\usepackage{pgfplotstable}
\pgfplotsset{compat=newest}
\usetikzlibrary{matrix,calc}
\usetikzlibrary{fit}
\usepackage{xfp}
\usepackage{mathtools}

\usetikzlibrary{positioning}

\usepackage{makecell}
%\usepackage{tabu}
\usepackage{tikz}
\usetikzlibrary{trees}



\usepgfplotslibrary{fillbetween}

\usepackage{filecontents}


\newtheorem{theorem}{Theorem}
\newtheorem{definition}{Definition}
\newcommand{\eat}[1]{}

\definecolor{bluegreen}{RGB}{3, 166, 155}
\definecolor{pitchblack}{RGB}{0, 0, 0}
\definecolor{lightbeige}{RGB}{255, 251, 241}
\definecolor{mediumgray}{RGB}{183, 183, 183}
\definecolor{mygreen}{rgb}{0,0.6,0}
\definecolor{mygray}{rgb}{0.5,0.5,0.5}
\definecolor{mymauve}{rgb}{0.58,0,0.82}
\definecolor{keywords}{RGB}{255,0,90}
\definecolor{comments}{RGB}{0,0,113}
\definecolor{red}{RGB}{255,0,0}
\definecolor{green}{RGB}{0,255,0}
\definecolor{navy}{RGB}{0,0,128}
\definecolor{DarkGrenen}{RGB}{0,100,0}
\definecolor{DarkOliveGreen}{RGB}{85,107,47}
\definecolor{saddlebrown}{RGB}{139,69,19}
\definecolor{gold}{RGB}{252,194,1}
\definecolor{tug}{RGB}{247,1,70}
\definecolor{tugb}{RGB}{120,137,251}


\definecolor{blue0}{RGB}{153,153,153}
\definecolor{blue1}{RGB}{77,77,77}%
\definecolor{blue2}{RGB}{165,71,209}
\definecolor{blue3}{RGB}{77,10,142}
\definecolor{blue4}{RGB}{74,139,203}
\definecolor{blue5}{RGB}{40,40,190}

\definecolor{color1}{RGB}{100,149,237} % corn flower blue
\definecolor{color2}{RGB}{153,153,153} % light gray
\definecolor{color3}{RGB}{0,0,0} % black
\definecolor{color4}{RGB}{255,165,0} % orange
\definecolor{color5}{RGB}{255,69,0} % orange red
\definecolor{color6}{RGB}{77,77,77} % dark gray
\definecolor{color7}{RGB}{31,119,180}
\definecolor{color8}{RGB}{7,77,125}
\definecolor{color9}{RGB}{153,216,201}


% Enable this two commands when you want to extract diagrams in extra files, then run "make"
\usetikzlibrary{external}
\tikzexternalize[prefix=plots/] %  activate

\usetikzlibrary{positioning}

\sloppy
\clubpenalty = 10000
\widowpenalty = 10000
\brokenpenalty = 10000
\frenchspacing

\begin{document}

    \title{GIO's Plots}

    \author{Paper-ID: 000}


    \maketitle


    % \tikzsetnextfilename{Experiment1a10k}
    % \begin{figure*}[h]
    %     \centering
    %     \begin{tikzpicture}[scale=1]

    \newcommand{\myaddplotExpOne}[5]{
        \addplot[ybar,xshift=#3,draw=black,fill=#4,line width=0.15pt, discard if not={#1}{#2}]
        table[ y=time, col sep=comma, x=example_nrows]
            {../../results/Experiment1a10k_times.dat};
        \addlegendentry{#5};
    };

    \newcommand{\myaddplotIndex}[2]{
        \addplot[color=#1,mark=-,xshift=#2, line width=1pt, only marks,mark options={scale=1.3,solid}]
        table[ y=time, col sep=comma, x=example_nrows] {../../results/Experiment0a_times.dat};
        \label{ppfza}
    };

    \pgfplotsset{
        discard if/.style 2 args={
            x filter/.code={
                \edef\tempa{\thisrow{#1}}
                \edef\tempb{#2}
                \ifx\tempa\tempb
                \def\pgfmathresult{inf}
                \fi
            }
        },
        discard if not/.style 2 args={
            x filter/.code={
                \edef\tempa{\thisrow{dataset}}
                \edef\tempb{#1}
                \ifx\tempa\tempb
                \edef\tempc{\thisrow{query}}
                \edef\tempd{#2}
                \ifx\tempc\tempd
                    \edef\tempe{\thisrow{parallel}}
                    \edef\tempf{true}
                    \ifx\tempe\tempf
                    \else
                    \def\pgfmathresult{inf}
                    \fi        
                \else
                \def\pgfmathresult{inf}
                \fi
                \else
                \def\pgfmathresult{inf}
                \fi
            }
        },
    };

    \begin{axis}
        [        
        bar width=5pt,
        ymin=0,
        scaled x ticks=false,
        y tick label style={/pgf/number format/1000 sep={}},
        x tick label style={/pgf/number format/1000 sep={}},
        scaled y ticks=false,
        enlarge y limits={0.25,upper},
        enlarge x limits=0.05,
        ylabel={Execution Time[s]},
        xlabel={$\#$Sample Records},
        ytick={0,2000,4000,6000, 8000},
        yticklabels={0,2,4,6,8},
        xtick pos=left,
        ytick pos=left,
        yticklabel style = {font=\LARGE},
        ylabel style = {font=\LARGE, xshift=-4pt},
        xticklabel style = {font=\LARGE},
        xtick=data,
        xtick={1000,2000,3000,4000,5000,6000,7000,8000,9000,10000},
        xticklabels={1k,2k,3k,4k,5k,6k,7k,8k,9k,10k},
        % symbolic x coords={Q1,Q2,Q3,Q4,Q5},
        xlabel style = {font=\LARGE, yshift=0pt},
        height=0.63\columnwidth,
        width=1.3\columnwidth,
        nodes near coords,
        every node near coord/.style={font=\fontsize{0.1pt}{0.1}, rotate=0},
        legend image code/.code={\draw [#1] (0cm,-0.1cm) rectangle (0.20cm,0.25cm); },
        legend style = {
            font=\LARGE,
            draw=none,
            legend columns = -1,
            legend cell align={left},
        },
        legend pos = {north west}
        ]

        \myaddplotExpOne{aminer-author-json}{Q1}{-7.5pt}{color1}{Q1};
        \myaddplotExpOne{aminer-author-json}{Q2}{-2.5pt}{color2}{Q2};
        \myaddplotExpOne{aminer-author-json}{Q3}{2.5pt}{color4}{Q3};
        \myaddplotExpOne{aminer-author-json}{Q4}{7.5pt}{color5}{Q4};
        \myaddplotIndex{color3}{-7.5pt};
        \myaddplotIndex{color3}{-2.5pt};
        \myaddplotIndex{color3}{2.5pt};
        \myaddplotIndex{color3}{7.5pt};


        \node [draw=none,inner sep=0, font=\LARGE, anchor=west] (leg1) at (rel axis cs: 0.58,0.9) {\Huge{---} \LARGE{Raw Load \& Index} };

    \end{axis}
\end{tikzpicture}

    %     \caption{Experiment1a: Identification and Generate Readers Part(AMiner-Author-JSON)}
    % \end{figure*}

    % \tikzsetnextfilename{Experiment1b10k}
    % \begin{figure*}[h]
    %     \centering
    %     \input{figures/Experiment1b10k}
    %     \caption{Experiment1b: Identification and Generate Readers Part(AMiner-Paper-JSON)}
    % \end{figure*}

    % \tikzsetnextfilename{Experiment1c10k}
    % \begin{figure*}[h]
    %     \centering
    %     \input{figures/Experiment1c10k}
    %     \caption{Experiment1c: Identification and Generate Readers Part(Yelp-JSON)}
    % \end{figure*} 

    % \tikzsetnextfilename{Experiment1d10k}
    % \begin{figure*}[h]
    %     \centering
    %     \input{figures/Experiment1d10k}
    %     \caption{Experiment1a: Identification and Generate Readers Part(AMiner-Author-Orig-Data)}
    % \end{figure*}

    % \tikzsetnextfilename{Experiment1e10k}
    % \begin{figure*}[h]
    %     \centering
    %     \begin{tikzpicture}[scale=1]

    \newcommand{\myaddplotExpOne}[5]{
        \addplot[ ybar,xshift=#3,draw=black,fill=#4,line width=0.15pt, discard if not={#1}{#2}]
        table[ y=time, col sep=comma, x=example_nrows]
            {../../results/Experiment1e10k_times.dat};
        \addlegendentry{#5};
    };
    \newcommand{\myaddplotIndex}[2]{
        \addplot[color=#1,mark=-,xshift=#2, line width=1pt, only marks,mark options={scale=1.3,solid}]
        table[ y=time, col sep=comma, x=example_nrows] {../../results/Experiment0e_times.dat};
        \label{ppfza}
    };

    \pgfplotsset{
        discard if/.style 2 args={
            x filter/.code={
                \edef\tempa{\thisrow{#1}}
                \edef\tempb{#2}
                \ifx\tempa\tempb
                \def\pgfmathresult{inf}
                \fi
            }
        },
        discard if not/.style 2 args={
            x filter/.code={
                \edef\tempa{\thisrow{dataset}}
                \edef\tempb{#1}
                \ifx\tempa\tempb
                \edef\tempc{\thisrow{query}}
                \edef\tempd{#2}
                \ifx\tempc\tempd
                    \edef\tempe{\thisrow{parallel}}
                    \edef\tempf{true}
                    \ifx\tempe\tempf
                    \else
                    \def\pgfmathresult{inf}
                    \fi      
                \else
                \def\pgfmathresult{inf}
                \fi
                \else
                \def\pgfmathresult{inf}
                \fi
            }
        },
    };

    \begin{axis}
        [       
        bar width=5pt,
        scaled x ticks=false,
        ymin=0,
        y tick label style={/pgf/number format/1000 sep={}},
        x tick label style={/pgf/number format/1000 sep={}},
        scaled y ticks=false,
        enlarge y limits={0.25,upper},
        enlarge x limits=0.05,
        ylabel={Execution Time[s]},
        xlabel={$\#$Sample Records},
        ytick={0,20000,40000,60000,80000,100000},
        yticklabels={0,20,40,60,80,100},
        xtick pos=left,
        ytick pos=left,
        yticklabel style = {font=\LARGE},
        ylabel style = {font=\LARGE, xshift=-4pt},
        xticklabel style = {font=\LARGE},
        xtick=data,
        xtick={1000,2000,3000,4000,5000,6000,7000,8000,9000,10000},
        xticklabels={1k,2k,3k,4k,5k,6k,7k,8k,9k,10k},
        xlabel style = {font=\LARGE, yshift=0pt},
        height=0.63\columnwidth,
        width=1.3\columnwidth,
        nodes near coords,
        every node near coord/.style={font=\fontsize{0.1pt}{0.1}, rotate=0},
        legend image code/.code={\draw [#1] (0cm,-0.1cm) rectangle (0.20cm,0.25cm); },
        legend style = {
            font=\LARGE,
            draw=none,
            legend columns = -1,
            legend cell align={left},
        },
        legend pos = {north west}
        ]

        \myaddplotExpOne{aminer-paper}{Q1}{-7.5pt}{color1}{Q18};
        \myaddplotExpOne{aminer-paper}{Q2}{-2.5pt}{color2}{Q19};
        \myaddplotExpOne{aminer-paper}{Q3}{2.5pt}{color4}{Q20};
        \myaddplotExpOne{aminer-paper}{Q4}{7.5pt}{color5}{Q21};

        \myaddplotIndex{color3}{-7.5pt};
        \myaddplotIndex{color3}{-2.5pt};
        \myaddplotIndex{color3}{2.5pt};
        \myaddplotIndex{color3}{7.5pt};
        \node [draw=none,inner sep=0, font=\LARGE, anchor=west] (leg1) at (rel axis cs: 0.58,0.9) {\Huge{---} \LARGE{Raw Load \& Index} };


    \end{axis}
\end{tikzpicture}

    %     \caption{Experiment1b: Identification and Generate Readers Part(AMiner-Paper-Orig-Data)}
    % \end{figure*}

    % \tikzsetnextfilename{Experiment1f10k}
    % \begin{figure}[h]
    %     \centering
    %     \begin{tikzpicture}[scale=1]

    \newcommand{\myaddplotExpOne}[5]{
        \addplot[ybar,xshift=#3,draw=black,fill=#4,line width=0.15pt, discard if not={#1}{#2}]
        table[ y=time, col sep=comma, x=example_nrows]
            {../../results/Experiment1f10k_times.dat};
        \addlegendentry{#5};
    };
    \newcommand{\myaddplotIndex}[2]{
        \addplot[color=#1,mark=-,xshift=#2, line width=1pt, only marks,mark options={scale=1,solid}]
        table[ y=time, col sep=comma, x=example_nrows] {../../results/Experiment0f_times.dat};
        \label{ppfza}
    };

    \pgfplotsset{
        discard if/.style 2 args={
            x filter/.code={
                \edef\tempa{\thisrow{#1}}
                \edef\tempb{#2}
                \ifx\tempa\tempb
                \def\pgfmathresult{inf}
                \fi
            }
        },
        discard if not/.style 2 args={
            x filter/.code={
                \edef\tempa{\thisrow{dataset}}
                \edef\tempb{#1}
                \ifx\tempa\tempb
                \edef\tempc{\thisrow{query}}
                \edef\tempd{#2}
                \ifx\tempc\tempd
                    \edef\tempe{\thisrow{parallel}}
                    \edef\tempf{true}
                    \ifx\tempe\tempf
                    \else
                    \def\pgfmathresult{inf}
                    \fi      
                \else
                \def\pgfmathresult{inf}
                \fi
                \else
                \def\pgfmathresult{inf}
                \fi
            }
        },
    };

    \begin{axis}
        [       
        bar width=4pt,
        scaled x ticks=false,
        ymin=0,
        y tick label style={/pgf/number format/1000 sep={}},
        x tick label style={/pgf/number format/1000 sep={}},
        scaled y ticks=false,
        enlarge y limits={0.25,upper},
        enlarge x limits=0.08,
        ylabel={Execution Time[s]},
        xlabel={$\#$Sample Records},
        ytick={0,2000,4000,6000},
        yticklabels={0,2,4,6},
        xtick pos=left,
        ytick pos=left,
        yticklabel style = {font=\LARGE},
        ylabel style = {font=\LARGE, xshift=-4pt},
        xticklabel style = {font=\LARGE},
        xtick=data,
        xtick={1000,2000,3000,4000,5000,6000,7000,8000,9000,10000},
        xticklabels={1k,,3k,,5k,,7k,,9k,},
        xlabel style = {font=\LARGE, yshift=0pt},
        height=0.63\columnwidth,
        width=0.68\columnwidth,
        nodes near coords,
        every node near coord/.style={font=\fontsize{0.1pt}{0.1}, rotate=0},
        legend image code/.code={\draw [#1] (0cm,-0.1cm) rectangle (0.20cm,0.25cm); },
        legend style = {
            font=\LARGE,
            draw=none,
            legend columns = -1,
            legend cell align={left},
        },
        legend pos = {north west}
        ]

        \myaddplotExpOne{yelp-csv}{Q1}{-2.5pt}{color1}{Q22};
        \myaddplotExpOne{yelp-csv}{Q2}{2.5pt}{color4}{Q23};

        \myaddplotIndex{color3}{-2.5pt};
        \myaddplotIndex{color3}{2.5pt};        
        \node [draw=none,inner sep=0, font=\LARGE, anchor=west] (leg1) at (rel axis cs: 0,0.75) {\Huge{--} \LARGE{Raw Load \& Ind.} };


    \end{axis}
\end{tikzpicture}

    %     \caption{Experiment1d: Identification and Generate Readers Part(Yelp-CSV)}
    % \end{figure}


    % \tikzsetnextfilename{Experiment1g10k}
    % \begin{figure}[h]
    %     \centering
    %     \input{figures/Experiment1g10k.tex}
    %     \caption{Experiment1g: Identification and Generate Readers Part(Message-HL7)}
    % \end{figure}

    % %%%%%%%%%%%%%%%%%%%%%%%%%%%%%%%%%%%%%%%%%%%%%%%%%%%%%%%%%%%%%%%%%%%%%%%%%%%%%%%%%%%%%%%%%%%%%%    

    % \tikzsetnextfilename{Experiment1a}
    % \begin{figure}[h]
    %     \centering
    %     \input{figures/Experiment1a.tex}
    %     \caption{Experiment1a: Readers Performance(AMiner-Author(JSON))}
    % \end{figure}

    % \tikzsetnextfilename{Experiment1b}
    % \begin{figure}[h]
    %     \centering
    %     \input{figures/Experiment1b.tex}
    %     \caption{Experiment1b: Readers Performance(AMiner-Paper(JSON))}
    % \end{figure}

    % \tikzsetnextfilename{Experiment1c}
    % \begin{figure}[h]
    %     \centering
    %     \input{figures/Experiment1c.tex}
    %     \caption{Experiment1c: Readers Performance(Yelp(JSON))}
    % \end{figure}

    % \tikzsetnextfilename{Experiment1d}
    % \begin{figure}[h]
    %     \centering
    %     \begin{tikzpicture}[scale=1]

    \newcommand{\myaddplotExpOne}[5]{
        \addplot[xshift=#3,draw=black,fill=#4,line width=0.15pt, discard if not={#1}{#2}]
        table[ y=time, col sep=comma, x=example_nrows]
            {../../results/Experiment1d_times.dat};
        \addlegendentry{#5};
    };

    \pgfplotsset{
        discard if/.style 2 args={
            x filter/.code={
                \edef\tempa{\thisrow{#1}}
                \edef\tempb{#2}
                \ifx\tempa\tempb
                \def\pgfmathresult{inf}
                \fi
            }
        },
        discard if not/.style 2 args={
            x filter/.code={
                \edef\tempa{\thisrow{dataset}}
                \edef\tempb{#1}
                \ifx\tempa\tempb
                \edef\tempc{\thisrow{query}}
                \edef\tempd{#2}
                \ifx\tempc\tempd
                    \edef\tempe{\thisrow{parallel}}
                    \edef\tempf{true}
                    \ifx\tempe\tempf
                    \else
                    \def\pgfmathresult{inf}
                    \fi        
                \else
                \def\pgfmathresult{inf}
                \fi
                \else
                \def\pgfmathresult{inf}
                \fi
            }
        },
    };

    \begin{axis}
        [
        ybar,
        bar width=7pt,
        ymin=0,
        y tick label style={/pgf/number format/1000 sep={}},
        x tick label style={/pgf/number format/1000 sep={}},
        scaled y ticks=false,
        enlarge y limits={0.25,upper},
        enlarge x limits=0.08,
        ylabel={Execution Time[s]},
        xlabel={$\#$Sample Records},
        ytick={0,1000,2000,3000},
        yticklabels={0,1,2,3},
        xtick pos=left,
        ytick pos=left,
        yticklabel style = {font=\Huge},
        ylabel style = {font=\Huge, xshift=-4pt},
        xticklabel style = {font=\Huge},
        xtick=data,
        xtick={200,300,400,500,600,700,800,900,1000},
        % symbolic x coords={Q1,Q2,Q3,Q4,Q5},
        xlabel style = {font=\Huge, yshift=0pt},
        height=0.8\columnwidth,
        width=1.6\columnwidth,
        nodes near coords,
        every node near coord/.style={font=\fontsize{0.1pt}{0.1}, rotate=0},
        legend image code/.code={\draw [#1] (0cm,-0.2cm) rectangle (0.3cm,0.35cm); },
        legend style = {
            font=\Huge,
            draw=none,
            legend columns = -1,
            legend cell align={left},
        },
        legend pos = {north west}
        ]

        \myaddplotExpOne{aminer-author}{Q1}{2.7pt}{color1}{Q14\ \ };
        \myaddplotExpOne{aminer-author}{Q2}{0.9pt}{color2}{Q15\ \ };
        \myaddplotExpOne{aminer-author}{Q3}{-0.9pt}{color3}{Q16\ \ };
        \myaddplotExpOne{aminer-author}{Q4}{-2.7pt}{color4}{Q17\ \ };

    \end{axis}
\end{tikzpicture}

    %     \caption{Experiment1d: Readers Performance(AMiner-Author(Custom))}
    % \end{figure}

    % \tikzsetnextfilename{Experiment1e}
    % \begin{figure}[h]
    %     \centering
    %     \input{figures/Experiment1e.tex}
    %     \caption{Experiment1e: Readers Performance(AMiner-Paper(Custom))}
    % \end{figure}

    % \tikzsetnextfilename{Experiment1f}
    % \begin{figure}[h]
    %     \centering
    %     \begin{tikzpicture}[scale=1]

    \newcommand{\myaddplotExpOne}[5]{
        \addplot[xshift=#3,draw=black,fill=#4,line width=0.15pt, discard if not={#1}{#2}]
        table[ y=time, col sep=comma, x=example_nrows]
            {../../results/Experiment1f_times.dat};
        \addlegendentry{#5};
    };

    \pgfplotsset{
        discard if/.style 2 args={
            x filter/.code={
                \edef\tempa{\thisrow{#1}}
                \edef\tempb{#2}
                \ifx\tempa\tempb
                \def\pgfmathresult{inf}
                \fi
            }
        },
        discard if not/.style 2 args={
            x filter/.code={
                \edef\tempa{\thisrow{dataset}}
                \edef\tempb{#1}
                \ifx\tempa\tempb
                \edef\tempc{\thisrow{query}}
                \edef\tempd{#2}
                \ifx\tempc\tempd
                    \edef\tempe{\thisrow{parallel}}
                    \edef\tempf{true}
                    \ifx\tempe\tempf
                    \else
                    \def\pgfmathresult{inf}
                    \fi      
                \else
                \def\pgfmathresult{inf}
                \fi
                \else
                \def\pgfmathresult{inf}
                \fi
            }
        },
    };

    \begin{axis}
        [
        ybar,
        bar width=5.9pt,
        ymin=0,
        y tick label style={/pgf/number format/1000 sep={}},
        x tick label style={/pgf/number format/1000 sep={}},
        scaled y ticks=false,
        enlarge y limits={0.25,upper},
        enlarge x limits=0.08,
        ylabel={Execution Time[s]},
        xlabel={$\#$Sample Records},
        ytick={0,500,1000,1500,2000,2500},
        yticklabels={0,0.5,1,1.5,2,2.5},
        xtick pos=left,
        ytick pos=left,
        yticklabel style = {font=\Huge},
        ylabel style = {font=\Huge, xshift=-4pt},
        xticklabel style = {font=\Huge},
        xtick=data,
        xtick={200,300,400,500,600,700,800,900,1000},
        xticklabels={200,,400,,600,,800,,1000},
        % symbolic x coords={Q1,Q2,Q3,Q4,Q5},
        xlabel style = {font=\Huge, yshift=0pt},
        height=0.8\columnwidth,
        width=0.83\columnwidth,
        nodes near coords,
        every node near coord/.style={font=\fontsize{0.1pt}{0.1}, rotate=0},
        legend image code/.code={\draw [#1] (0cm,-0.2cm) rectangle (0.30cm,0.25cm); },
        legend style = {
            font=\Huge,
            draw=none,
            legend columns = -1,
            legend cell align={left},
        },
        legend pos = {north west}
        ]

        \myaddplotExpOne{yelp-csv}{Q1}{0pt}{color1}{Q22\ \ };
        \myaddplotExpOne{yelp-csv}{Q2}{-1.4pt}{color6}{Q23\ \ };

    \end{axis}
\end{tikzpicture}

    %     \caption{Experiment1f: Readers Performance(Yelp(CSV))}
    % \end{figure}

    % \tikzsetnextfilename{Experiment1g}
    % \begin{figure}[h]
    %     \centering
    %     \input{figures/Experiment1g.tex}
    %     \caption{Experiment1g: Readers Performance(HL7(Custom))}
    % \end{figure}

    %%%%%%%%%%%%%%%%%%%%%%%%%%%%%%%%%%%%%%%%%%%%%%%%%%%%%%%%%%%%%%%%%%%%%%%%%%%%%%%%%%%%%%%%%%%%%%%%%%%%%%%%%%
    % \tikzsetnextfilename{Experiment2a}
    % \begin{figure}[h]
    %     \centering
    %     
 \newcommand{\myaddplot}[6]{
	\addplot[xshift=#3,draw=black,line width=0.15pt, fill=#4, discard if single={#1}{#2}{#5}{#6}]
	table[ y=time, col sep=comma, x=query] {../../results/Experiment2a_times.dat};
	\label{pp#1}
};
\newcommand{\myaddplotidentify}[4]{
	\addplot[xshift=#2,draw=none, fill=black,line width=0.15pt, discard if notidentify={#1}{#3}{#4}]
	table[ y=time, col sep=comma, x=query] {../../results/Experiment1a_times.dat};
	\label{ppiGIO}
};


\newcommand{\myaddplotp}[6]{
	\addplot[xshift=#3,draw=black,line width=0.15pt, fill=#4, discard if single={#1}{#2}{#5}{#6}, postaction={pattern=north east lines,pattern color=black}]
	table[ y=time, col sep=comma, x=query] {../../results/Experiment2a_times.dat};
	%\label{pp#1}
};

\newcommand{\addDiagramExpThree}[6]{	
	\nextgroupplot[
		title style={font=\small, yshift=-7pt},		
		bar width=7pt,
		width=0.35\columnwidth,    	
		yticklabels=#5,
		xticklabels={#6,#6,#6,#6,#6}, 		
	]
	\myaddplot{GIO}{#1}{3.5pt}{tug}{#2}{#3};
	\myaddplotidentify{#1}{-3.5pt}{#2}{#3};
	\myaddplot{RapidJSON}{#1}{-3.5pt}{color1}{#2}{#3};
	\myaddplot{SystemDS+JACKSON}{#1}{-3.5pt}{color6}{#2}{#3};
	\myaddplot{SystemDS+GSON}{#1}{-3.5pt}{color4}{#2}{#3};
	\myaddplot{SystemDS+JSON4J}{#1}{-3.5pt}{color5}{#2}{#3};	
	\node [draw=none,inner sep=0, font=\LARGE, anchor=west,rotate=90](leg1) at (rel axis cs: 0.3,.63) {#4};	
};  

\newcommand{\addDiagramExpThreep}[6]{	
	\nextgroupplot[
		title style={font=\small, yshift=-7pt},		
		bar width=7pt,
		width=0.36\columnwidth,    	
		yticklabels=#5,
		xticklabels={#6,#6,#6,#6,#6}, 		
	]
	\myaddplotp{GIO}{#1}{3.5pt}{tug}{#2}{#3};
	\myaddplotidentify{#1}{-3.5pt}{#2}{#3};
	\myaddplotp{RapidJSON}{#1}{-3.5pt}{color1}{#2}{#3};
	\myaddplotp{SystemDS+JACKSON}{#1}{-3.5pt}{color6}{#2}{#3};
	\myaddplotp{SystemDS+GSON}{#1}{-3.5pt}{color4}{#2}{#3};
	\myaddplotp{SystemDS+JSON4J}{#1}{-3.5pt}{color5}{#2}{#3};	
	\node [draw=none,inner sep=0, font=\LARGE, anchor=west,rotate=90](leg1) at (rel axis cs: 0.3,.60) {#4};	
}; 

\pgfplotsset{
	discard if single/.style n args={4}{
		x filter/.code={
			\edef\tempa{\thisrow{baseline}}
			\edef\tempb{#1}
			\ifx\tempa\tempb
			\edef\tempc{\thisrow{dataset}}
			\edef\tempd{#2}
			\ifx\tempc\tempd
				\edef\tempe{\thisrow{parallel}}
				\edef\tempf{#3}
				\ifx\tempe\tempf
					\edef\tempg{\thisrow{query}}
					\edef\temph{#4}
					\ifx\tempg\temph
					\else
					\def\pgfmathresult{inf}
					\fi   
				\else
				\def\pgfmathresult{inf}
				\fi      
			\else
			\def\pgfmathresult{inf}
			\fi
			\else
			\def\pgfmathresult{inf}
			\fi
		}
	},
	discard if notidentify/.style n args={3}{
		x filter/.code={
			\edef\tempa{\thisrow{dataset}}
			\edef\tempb{#1}
			\ifx\tempa\tempb
			\edef\tempc{\thisrow{example_nrows}}
			\edef\tempd{200}
			\ifx\tempc\tempd
				\edef\tempe{\thisrow{parallel}}
				\edef\tempf{true}
				\ifx\tempe\tempf
					\edef\tempg{\thisrow{query}}
					\edef\temph{#3}
					\ifx\tempg\temph
					\else
					\def\pgfmathresult{inf}
					\fi 
				\else
				\def\pgfmathresult{inf}
				\fi      
			\else
			\def\pgfmathresult{inf}
			\fi
			\else
			\def\pgfmathresult{inf}
			\fi
		}
	}
};

\begin{tikzpicture}
	\begin{groupplot}[
	  group style={
		group size=8 by 1,
		x descriptions at=edge bottom,
		y descriptions at=edge left,
		horizontal sep=4pt,
		vertical sep=0pt
	  },
	  	axis y line*=left,
		axis x line*=bottom,
		xtick style={draw=none},		
	  	every major tick/.append style={ thick,major tick length=2.5pt, gray},
	 	axis line style={gray},
	  	ybar,        
      	ybar=0pt,
        ymin=0,
		ymax = 25000,
        y tick label style={/pgf/number format/1000 sep={}},
        x tick label style={/pgf/number format/1000 sep={}},
        scaled y ticks=false,
        enlarge y limits={0.1,upper},
        enlarge x limits=0,
        ylabel={Execution Time[s]},
        ytick={0,5000,10000,15000,20000,25000},
        yticklabels={0,5,10,15,20,25},
        ytick align=outside,
        xtick pos=left,
        ytick pos=left,
        yticklabel style = {font=\Huge},
        ylabel style = {font=\Huge, yshift=-1pt},
        xticklabel style = {font=\Huge, xshift=25pt},
        xtick=data,
		height=0.75\columnwidth,   
		ymajorgrids=true,
  		grid style=dotted,   
		minor grid style={gray!50},  
	  	symbolic x coords={Q1,Q2,Q3,Q4,Q5},      
		legend image code/.code={
            \draw [#1] (0cm,-0.1cm) rectangle (0.25cm,0.3cm); },			
        ]
	  ]	  
	  \addDiagramExpThree{aminer-author-json}{false}{Q1}{1-Thread}{{0,5,10,15,20,25}}{Q1};
	  \addDiagramExpThreep{aminer-author-json}{true}{Q1}{32-Thread}{{}}{}; 	 
	  \addDiagramExpThree{aminer-author-json}{false}{Q2}{1-Thread}{{}}{Q2};
	  \addDiagramExpThreep{aminer-author-json}{true}{Q2}{32-Thread}{{}}{};
	  \addDiagramExpThree{aminer-author-json}{false}{Q3}{1-Thread}{{}}{Q3};
	  \addDiagramExpThreep{aminer-author-json}{true}{Q3}{32-Thread}{{}}{};
	  \addDiagramExpThree{aminer-author-json}{false}{Q4}{1-Thread}{{}}{Q4};
	  \addDiagramExpThreep{aminer-author-json}{true}{Q4}{32-Thread}{{}}{};

	\end{groupplot}	
	\node[font=\huge] at ($(group c4r1.north west) + (1.5cm,0.4cm)$) {\shortstack[l]{
		\ref{ppGIO} GIO 
		\ref{ppiGIO} I/O Gen
		\ref{ppRapidJSON} RapidJSON
		\ref{ppSystemDS+JACKSON} Jackson 	    
	    \ref{ppSystemDS+GSON} Gson 
	    \ref{ppSystemDS+JSON4J} JSON4J
	}};	

	\draw[thick, black] ($(group c1r1.north west)+(0pt,0)$) -- ($(group c2r1.north east)+(0pt,0pt)$) -- 
	($(group c2r1.south east)+(0pt,0pt)$) -- ($(group c1r1.south west)+(0pt,0pt)$) -- ($(group c1r1.north west)+(0pt,0)$);

	\draw[thick, black] ($(group c3r1.north west)+(0pt,0)$) -- ($(group c4r1.north east)+(0pt,0pt)$) -- 
	($(group c4r1.south east)+(0pt,0pt)$) -- ($(group c3r1.south west)+(0pt,0pt)$) -- ($(group c3r1.north west)+(0pt,0)$);


	\draw[thick, black] ($(group c5r1.north west)+(0pt,0)$) -- ($(group c6r1.north east)+(0pt,0pt)$) -- 
	($(group c6r1.south east)+(0pt,0pt)$) -- ($(group c5r1.south west)+(0pt,0pt)$) -- ($(group c5r1.north west)+(0pt,0)$);

	\draw[thick, black] ($(group c7r1.north west)+(0pt,0)$) -- ($(group c8r1.north east)+(0pt,0pt)$) -- 
	($(group c8r1.south east)+(0pt,0pt)$) -- ($(group c7r1.south west)+(0pt,0pt)$) -- ($(group c7r1.north west)+(0pt,0)$);

	\draw[thick, black] ($(group c2r1.south west)+(0pt,0)$) -- ($(group c2r1.south west)+(0pt,-3pt)$);
	\draw[thick, black] ($(group c4r1.south west)+(0pt,0)$) -- ($(group c4r1.south west)+(0pt,-3pt)$);
	\draw[thick, black] ($(group c6r1.south west)+(0pt,0)$) -- ($(group c6r1.south west)+(0pt,-3pt)$);
	\draw[thick, black] ($(group c8r1.south west)+(0pt,0)$) -- ($(group c8r1.south west)+(0pt,-3pt)$);


	
  \end{tikzpicture}
    %     \caption{Experiment2a: Readers Performance(AMiner-Author(JSON))}
    % \end{figure}

    % \tikzsetnextfilename{Experiment2b}
    % \begin{figure}[h]
    %     \centering
    %     \input{figures/Experiment2b.tex}
    %     \caption{Experiment2b: Readers Performance(AMiner-Paper(JSON))}
    % \end{figure}

    % \tikzsetnextfilename{Experiment2c}
    % \begin{figure}[h]
    %     \centering
    %     \input{figures/Experiment2c.tex}
    %     \caption{Experiment2c: Readers Performance(Yelp(JSON))}
    % \end{figure}

    % \tikzsetnextfilename{Experiment2d}
    % \begin{figure}[h]
    %     \centering
    %     \input{figures/Experiment2d.tex}
    %     \caption{Experiment2d: Readers Performance(AMiner-Author(Custom))}
    % \end{figure}

    % \tikzsetnextfilename{Experiment2e}
    % \begin{figure}[h]
    %     \centering
    %     \input{figures/Experiment2e.tex}
    %     \caption{Experiment2e: Readers Performance(AMiner-Paper(Custom))}
    % \end{figure}

    % \tikzsetnextfilename{Experiment2f}
    % \begin{figure}[h]
    %     \centering
    %     \input{figures/Experiment2f.tex}
    %     \caption{Experiment2f: Readers Performance(Yelp(CSV))}
    % \end{figure}

    \tikzsetnextfilename{Experiment2g}
    \begin{figure}[h]
        \centering
        \input{figures/Experiment2g.tex}
        \caption{Experiment2g: Readers Performance(HL7(Custom))}
    \end{figure}
   

    % %%%%%%%%%%%%%%%%%%%%%%%%%%%%%%%%%%%%%%%%%%%%%%%%%%%%%%%%%%%%%%%%%%%%%%%%%%%%%%%%%%%%%%%%%%%%%%%%%%%%%%%%%
    % \tikzsetnextfilename{Experiment4a}
    % \begin{figure}[h]
    %     \centering
    %     \input{figures/Experiment4a.tex}
    %     \caption{Experiment4a: Readers Performance(AMiner-Author(JSON))}
    % \end{figure}

    % \tikzsetnextfilename{Experiment4b}
    % \begin{figure}[h]
    %     \centering
    %     \input{figures/Experiment4b.tex}
    %     \caption{Experiment4b: Readers Performance(AMiner-Paper(JSON))}
    % \end{figure}

    % \tikzsetnextfilename{Experiment4c}
    % \begin{figure}[h]
    %     \centering
    %     \input{figures/Experiment4c.tex}
    %     \caption{Experiment4c: Readers Performance(Yelp(JSON))}
    % \end{figure}

    % \tikzsetnextfilename{Experiment4d}
    % \begin{figure}[h]
    %     \centering
    %     \input{figures/Experiment4d.tex}
    %     \caption{Experiment4d: Readers Performance(AMiner-Author(Custom))}
    % \end{figure}

    % \tikzsetnextfilename{Experiment4e}
    % \begin{figure}[h]
    %     \centering
    %     \input{figures/Experiment4e.tex}
    %     \caption{Experiment4e: Readers Performance(AMiner-Paper(Custom))}
    % \end{figure}

    % \tikzsetnextfilename{Experiment4f}
    % \begin{figure}[h]
    %     \centering
    %     \input{figures/Experiment4f.tex}
    %     \caption{Experiment4f: Readers Performance(Yelp(CSV))}
    % \end{figure}

    % %HL7
    % \tikzsetnextfilename{Experiment4g}
    % \begin{figure}[h]
    %     \centering
    %     \input{figures/Experiment4g.tex}
    %     \caption{Experiment4g: Readers Performance(HL7))}
    % \end{figure}

    % \tikzsetnextfilename{Experiment4h}
    % \begin{figure}[h]
    %     \centering
    %     \input{figures/Experiment4h.tex}
    %     \caption{Experiment4h: Readers Performance(mnist8m-libsvm)}
    % \end{figure}


    % \tikzsetnextfilename{Experiment4i}
    % \begin{figure}[h]
    %     \centering
    %     \input{figures/Experiment4i.tex}
    %     \caption{Experiment4i: Readers Performance(Susy(libsvm))}
    % \end{figure}

    % \tikzsetnextfilename{Experiment4j}
    % \begin{figure}[h]
    %     \centering
    %     \begin{tikzpicture}[scale=1, node distance=6.0mm]
    \newcommand{\myaddplot}[5]{
        \addplot[color=#4,mark=#3,discard if single={#1}{#2}{#5}]
        table[ y=time, col sep=comma, x=field] {../../results/Experiment4j_times.dat};
        \label{ppfa#1}
    };
    \newcommand{\myaddplotidentifyFour}[3]{
        \addplot[color=#2,mark=triangle*, discard if notidentify={#1}{#3}]
        table[ y=time, col sep=comma, x=field] {../../results/Experiment3j_times.dat};
        \label{ppfaiGIO}
    };
    \newcommand{\addDiagramExpThree}[2]{       
        \myaddplotidentifyFour{#1}{color4}{#2};
        \myaddplot{Python}{#1}{*}{color1}{#2};
        \myaddplot{SystemDS+CSV}{#1}{square*}{color8}{#2};
        \myaddplot{GIO}{#1}{triangle*}{tug}{#2};

        \node [draw=none,inner sep=0, font=\LARGE, anchor=west](leg1) at (rel axis cs: 0,0.85) {\shortstack[l]{
            \ref{ppfaGIO} GIO \\ \\
            \ref{ppfaPython} Python
        }};

        \node [draw=none,inner sep=1, font=\LARGE, right=of leg1,xshift=-5.2mm]{\shortstack[l]{
            \ref{ppfaiGIO} I/O Gen \\ \\
            \ref{ppfaSystemDS+CSV} SystemDS
        }};
   };

   
   \pgfplotsset{
	discard if single/.style n args={3}{
		x filter/.code={
			\edef\tempa{\thisrow{baseline}}
			\edef\tempb{#1}
			\ifx\tempa\tempb
			\edef\tempc{\thisrow{dataset}}
			\edef\tempd{#2}
			\ifx\tempc\tempd
				\edef\tempe{\thisrow{parallel}}
				\edef\tempf{#3}
				\ifx\tempe\tempf
				\else
				\def\pgfmathresult{inf}
				\fi      
			\else
			\def\pgfmathresult{inf}
			\fi
			\else
			\def\pgfmathresult{inf}
			\fi
		}
	},
	discard if notidentify/.style n args={2}{
		x filter/.code={
			\edef\tempa{\thisrow{dataset}}
			\edef\tempb{#1}
			\ifx\tempa\tempb
			\edef\tempc{\thisrow{example_nrows}}
			\edef\tempd{200}
			\ifx\tempc\tempd
				\edef\tempe{\thisrow{parallel}}
				\edef\tempf{#2}
				\ifx\tempe\tempf					
				\else
				\def\pgfmathresult{inf}
				\fi      
			\else
			\def\pgfmathresult{inf}
			\fi
			\else
			\def\pgfmathresult{inf}
			\fi
		}
	}
};


    \begin{axis}
        [
        ymin=0,
        y tick label style={/pgf/number format/1000 sep={}},
        x tick label style={/pgf/number format/1000 sep={}},
        scaled y ticks=false,
        enlarge y limits={0.6,upper},
        enlarge x limits=0.009,
        ylabel={Execution Time[s]},
        xlabel={$\#$ Parsed Fields},
        ytick={0,10000,20000,30000,40000},
        yticklabels={0,10,20,30,40},
        ytick align=outside,
        xtick align=outside,
        xtick pos=left,
        ytick pos=left,
        yticklabel style = {font=\Huge},
        ylabel style = {font=\Huge},
        xticklabel style = {font=\Huge},
        xtick=data,
        symbolic x coords={F0,F1,F2,F3,F4,F5,F6,F7,F8,F9,F10,F11,F12,F13,F14,F15,F16,F17,F18,F19,F20,F21,F22,F23,F24,F25,F26,F27,F28, F29, F30, F31, F32},
        xticklabels={},
        extra x ticks={F0,F12,F22,F32},
        extra x tick labels={1,120,220,313},
        extra x tick style={major tick style={black, thick}},
        xlabel style = {font=\Huge, yshift=0pt},
        height=0.8\columnwidth,
        width=1.03\columnwidth,
        grid=both,
        grid style=dotted,
        minor grid style={gray!50},
        nodes near coords,
        %legend image post style={line width=1.5pt,scale=0.6},
        every node near coord/.style={font=\fontsize{0.1pt}{0.1}, rotate=0},
        every axis plot/.append style={line width=0.9pt,mark options={scale=1,solid}},
        ]
        \addDiagramExpThree{ReWasteF-csv}{true};
    \end{axis}

\end{tikzpicture}

    %     \caption{Experiment4i: Readers Performance(ReWasteF(CSV))}
    % \end{figure}


    % \tikzsetnextfilename{Experiment4k}
    % \begin{figure}[h]
    %     \centering
    %     \input{figures/Experiment4k.tex}
    %     \caption{Experiment4i: Readers Performance(Higgs(CSV))}
    % \end{figure}


    % \tikzsetnextfilename{Experiment4l}
    % \begin{figure}[h]
    %     \centering
    %     \input{figures/Experiment4l.tex}
    %     \caption{Experiment4i: Readers Performance(Queen(MM))}
    % \end{figure}

    %  \tikzsetnextfilename{Experiment4m}
    % \begin{figure}[h]
    %     \centering
    %     \input{figures/Experiment4m.tex}
    %     \caption{Experiment4m: Readers Performance(Relat9(MM))}
    % \end{figure}


    % \tikzsetnextfilename{Experiment5a}
    % \begin{figure}[h]
    %     \centering
    %     \begin{tikzpicture}
	\newcommand{\myaddplot}[5]{
	   \addplot[xshift=#1,draw=black,line width=0.15pt, fill=#2, discard if single={#3}{#4}{#5}] 
	   table[ y=time, col sep=comma, x=dataset] {../../results/Experiment5b_times.dat};
	   \label{pp#3}
   };
   
   \newcommand{\myaddplotGIO}[2]{
	   \addplot[xshift=#1,draw=black,line width=0.15pt, fill=#2, discard if singlegio={GIO}] 
	   table[ y=time, col sep=comma, x=dataset] {../../results/Experiment5b_times.dat};
	   \label{ppGIO}
   };
   
   \newcommand{\myaddplotidentify}[2]{
	   \addplot[xshift=#1,draw=none, fill=black,line width=0.15pt, discard if notidentify={#2}]
	   table[ y=time, col sep=comma, x=dataset] {../../results/Experiment5a_times.dat};
	   \label{ppiGIO}
   };
   
   \pgfplotsset{
	   discard if single/.style n args={3}{
		   x filter/.code={
			   \edef\tempa{\thisrow{baseline}}
			   \edef\tempb{#1}
			   \ifx\tempa\tempb
					   \edef\tempe{\thisrow{parallel}}
					   \edef\tempf{#2}
					   \ifx\tempe\tempf
						   %%%%%%%%%%
							   \edef\tempg{\thisrow{dataset}}
							   \edef\temph{#3}
							   \ifx\tempg\temph				
							   \else
							   \def\pgfmathresult{inf}
							   \fi      
						   %%%%%%%%%%					
					   \else
					   \def\pgfmathresult{inf}
					   \fi      
			   \else
			   \def\pgfmathresult{inf}
			   \fi			
		   }
	   },
	   discard if singlegio/.style n args={1}{
		   x filter/.code={
			   \edef\tempa{\thisrow{baseline}}
			   \edef\tempb{#1}
			   \ifx\tempa\tempb
					   \edef\tempe{\thisrow{parallel}}
					   \edef\tempf{false}
					   \ifx\tempe\tempf						
					   \else
					   \def\pgfmathresult{inf}
					   \fi      
			   \else
			   \def\pgfmathresult{inf}
			   \fi			
		   }
	   },
	   discard if notidentify/.style n args={1}{
		   x filter/.code={
			   \edef\tempa{\thisrow{parallel}}
			   \edef\tempb{#1}
			   \ifx\tempa\tempb
			   \else
			   \def\pgfmathresult{inf}
			   \fi
		   }
	   }
   };
   
   
	   \begin{axis}[
		   ymode=log,
		   xtick style={draw=none},		
			 every major tick/.append style={ thick,major tick length=2.5pt, gray},
			axis line style={gray},
			 ybar,        
			 ybar=0pt,
		   ymin=1,
		   log ticks with fixed point,
		   %ymax = 1700000,
		   y tick label style={/pgf/number format/1000 sep={}},
		   x tick label style={/pgf/number format/1000 sep={}},
		   scaled y ticks=false,
		   enlarge y limits={0.4,upper},
		   enlarge x limits=0.2,
		   ylabel={Execution Time[s]},
		   ytick={1,10,100,1000,10000},
           yticklabels={0,10,100,1e3,1e4},
		   ytick align=outside,
		   xtick pos=left,
		   ytick pos=left,
		   yticklabel style = {font=\Huge},
		   ylabel style = {font=\Huge, yshift=-2pt},
		   xticklabel style = {font=\Huge, xshift=2pt},
		   xtick=data,
		   height=0.9\columnwidth,  
		   bar width=17pt,
		   width=1.3\columnwidth,   
		   ymajorgrids=true,
			 grid style=dotted,   
		   minor grid style={gray!50},  
			 symbolic x coords={aminer-author,aminer-paper, message-hl7, autolead-xml},   
		   xticklabels={Author,Paper, HL7, ADF},    
		   legend image code/.code={
			   \draw [#1] (0cm,-0.1cm) rectangle (0.4cm,0.3cm); },			
		   ]	  	  
		\myaddplotGIO{40pt}{tug};
		\myaddplotidentify{23pt}{false};
		\myaddplot{23pt}{color1!50}{SystemDS}{false}{aminer-author};
		\myaddplot{6pt}{color1!50}{SystemDS}{false}{aminer-paper};
		\myaddplot{-11pt}{color2}{Python}{false}{message-hl7};
		\myaddplot{-11pt}{color4}{SystemDS+HAPI-HL7}{false}{message-hl7};
		\myaddplot{-45pt}{color7}{SystemDS+Jackson}{false}{autolead-xml};

		\node [draw=none,inner sep=0, font=\Huge, anchor=west] (leg1) at (rel axis cs: 0.02,0.72) {\shortstack[l]{
			\ref{ppGIO} GIO 
			\ref{ppiGIO} I/O Gen \\ 
			\ref{ppSystemDS} Java Hand-coded  \\ 
			\ref{ppPython} Python-HL7\\ 
			\ref{ppSystemDS+HAPI-HL7} HAPI-HL7\\ 
			\ref{ppSystemDS+Jackson} Jackson (XML) 			
	}};
	   \end{axis}		
	 \end{tikzpicture}
    %     \caption{Experiment5a: Readers Performance(EtoE)}
    % \end{figure}


    % \tikzsetnextfilename{Experiment5b}
    % \begin{figure}[h]
    %     \centering
    %     \input{figures/Experiment5b.tex}
    %     \caption{Experiment5b: Readers Performance(EtoE)}
    % \end{figure}

    % \tikzsetnextfilename{Experiment6a}
    % \begin{figure}[h]
    %     \centering
    %     \begin{tikzpicture}[scale=1]

    \newcommand{\myaddplotExpOne}[5]{
        \addplot[mark=#3,color=#4,line width=1pt, discard if not={#1}{#2}]
        table[ y=time, col sep=comma, x=example_nrows]
            {../../results/Experiment6a_times.dat};
        \addlegendentry{#5};
    };

    %\addplot[color=#4,mark=#3,discard if single={#1}{#2}{#5}]
    %    table[ y=time, col sep=comma, x=field] {../../results/Experiment4f_times.dat};
    %    \label{ppfa#1}

    \pgfplotsset{
        discard if/.style 2 args={
            x filter/.code={
                \edef\tempa{\thisrow{#1}}
                \edef\tempb{#2}
                \ifx\tempa\tempb
                \def\pgfmathresult{inf}
                \fi
            }
        },
        discard if not/.style 2 args={
            x filter/.code={
                \edef\tempa{\thisrow{dataset}}
                \edef\tempb{#1}
                \ifx\tempa\tempb
                \edef\tempc{\thisrow{baseline}}
                \edef\tempd{#2}
                \ifx\tempc\tempd
                    \edef\tempe{\thisrow{parallel}}
                    \edef\tempf{true}
                    \ifx\tempe\tempf
                    \else
                    \def\pgfmathresult{inf}
                    \fi        
                \else
                \def\pgfmathresult{inf}
                \fi
                \else
                \def\pgfmathresult{inf}
                \fi
            }
        },
    };

    \begin{axis}
        [
        %ybar,
        ymode=log,        
        %bar width=5.5pt,
        ymin=1000,
        y tick label style={/pgf/number format/1000 sep={}},
        x tick label style={/pgf/number format/1000 sep={}},
        scaled y ticks=false,
        enlarge y limits={0.4,upper},
        enlarge x limits=0.08,
        ylabel={Execution Time[s]},
        xlabel={$\#$Sample Records},
        ytick={1000,10000,100000,1000000,10000000},
        yticklabels={0,$10$,$10^2$,$10^3$, $10^4$},
        xtick pos=left,
        ytick pos=left,
        yticklabel style = {font=\large},
        ylabel style = {font=\large, xshift=-4pt},
        xticklabel style = {font=\large},
        xtick=data,
        xtick={100,200,300,400,500,600,700,800,900,1000},
        %xticklabels={100,,300,,500,,700,,900,},
        xticklabels={,200,,400,,600,,800,,1000},
        % symbolic x coords={Q1,Q2,Q3,Q4,Q5},
        xlabel style = {font=\large, yshift=0pt},
        height=0.48\columnwidth,
        width=0.7\columnwidth,
        nodes near coords,
        every node near coord/.style={font=\fontsize{0.1pt}{0.1}, rotate=0},
        %legend image code/.code={\draw [#1] (0cm,-0.1cm) rectangle (0.20cm,0.25cm); },
        legend style = {
            font=\large,
            draw=none,
            legend columns = -1,
            legend cell align={left},
        },
        legend pos = {north west}
        ]

        \myaddplotExpOne{mm-col}{GIO}{*}{tug}{GIO};
        \myaddplotExpOne{mm-col}{OLDGIO}{square*}{color1}{Early GIO};
        
    \end{axis}
\end{tikzpicture}

    %     \caption{Experiment6a: Readers Performance Old and New}
    % \end{figure}

    % \tikzsetnextfilename{Experiment6b}
    % \begin{figure}[h]
    %     \centering
    %     \input{figures/Experiment6b.tex}
    %     \caption{Experiment6b: Readers Performance Old and New}
    % \end{figure}

\end{document}
\endinput
