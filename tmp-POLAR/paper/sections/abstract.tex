%1. State the problem
Join ordering and query optimization are crucial for query performance but remain challenging due to unknown or changing characteristics of query intermediates, especially for complex queries with many joins.
%2. Say why it's an interesting problem
Over the past two decades, a spectrum of techniques for adaptive query processing (AQP)---including inter-/intra-operator adaptivity and tuple routing---have been proposed to address these challenges. However, commercial database systems in practice do not implement holistic AQP techniques because they increase the system complexity (e.g., intertwined planning and execution) and thus complicate debugging and testing. Additionally, existing approaches may incur large overheads, leading to problematic performance regressions.
%3. Say what your solution achieves
In this paper, we introduce POLAR, a simple yet very effective technique for a self-regulating selection of alternative join orderings with bounded overhead. We enhance left-deep join pipelines with alternative join orders, perform regret-bounded tuple routing to find and validate \enquote{plans of least resistance}, and then process the majority of tuple batches through these plans. We study different join order selection techniques, different routing strategies, and a variety of workload characteristics.
%4. Say what follows from your solution
Our experiments with a POLAR prototype on DuckDB show runtime improvements of up to 9x compared to non-adaptive pipelines and up to 74x compared to state-of-the-art AQP systems. With tuned exploration budgets, POLAR shows no more than 5\% overhead compared to DuckDB. 
